
\documentclass[12pt-letter paper]{article}              
\usepackage{siunitx}                                    
\usepackage{setspace}                                  
\usepackage{gensymb}                                    
\usepackage{xcolor}                                    
\usepackage{caption}                                  
%\usepackage{subcaption}                                
\doublespacing                                          
\singlespacing                                      
\usepackage[none]{hyphenat}                            
\usepackage{amssymb}                                    
\usepackage{relsize}                                
\usepackage[cmex10]{amsmath}                            
\usepackage{mathtools}                                  
\usepackage{amsmath}                              
\usepackage{commath}                                    
\usepackage{amsthm}                                    
\interdisplaylinepenalty=2500                    
%\savesymbol{iint}                                      
\usepackage{txfonts}                                    
%\restoresymbol{TXF}{iint}                      
\usepackage{wasysym}                                    
\usepackage{amsthm}
\usepackage{mathrsfs}                                  
\usepackage{txfonts}                                    
\let\vec\mathbf{}                                      
\usepackage{stfloats}
\usepackage{float}                                      
\usepackage{cite}                                      
\usepackage{cases}                                      
\usepackage{subfig}                                    
%\usepackage{xtab}                                      
\usepackage{longtable}
\usepackage{multirow}                                  
%\usepackage{algorithm}
\usepackage{amssymb}                                    
%\usepackage{algpseudocode}                            
\usepackage{enumitem}                                  
\usepackage{mathtools}                                  
%\usepackage{eenrc}                                    
%\usepackage[framemethod=tikz]{mdframed}              
\usepackage{listings}                                  
%\usepackage{listings}
\usepackage[latin1]{inputenc}                          
%%\usepackage{color}{                                  
%%\usepackage{lscape}                                  
\usepackage{textcomp}                                  
\usepackage{titling}                                    
\usepackage{hyperref}                                
%\usepackage{fulbigskip}                                
\usepackage{tikz}                                      
\usepackage{graphicx}                                
\lstset{                                                  
  frame=single,                                          
  breaklines=true                                      
}                                                      
\let\vec\mathbf{}                                      
\usepackage{enumitem}                              
\usepackage{graphicx}                                  
\usepackage{siunitx}                                    
\let\vec\mathbf{}                                      
\usepackage{enumitem}                                  
\usepackage{graphicx}                                  
\usepackage{enumitem}
\usepackage{tfrupee}                                    
\usepackage{amsmath}                                    
\usepackage{amssymb}                                  
\usepackage{mwe} % for blindtext and example-image-a in example                                                
\usepackage{wrapfig}            
\graphicspath{{figs/}}
\providecommand{\mydet}[1]{\ensuremath{\begin{vmatrix}#1
\end{\vmatrix}}}                                        
\providecommand{\myvec}[1]{\ensuremath{\begin{bmatrix}#1
\end{\bmatrix}}}                                      
\providecommand{\cbrak}[1]{\ensuremath{\left\{#1\right\}}}                                                      
\providecommand{\brak}[1]{\ensuremath{\left(#1\right)}}
\begin{document}                                      
\begin{center}                                    
\section*{SECETION-A}        
\end{center}    
\begin{enumerate}                                                                        
\item Which of the following quadratic equations has a sum of its roots as  $4$ ?:  
\begin{enumerate}
\item $ 2x^2 - 4x + 8 = 0 $                          
\item $ -x^2 - 4x + 4 = 0 $                          
\item $ \sqrt{2}x^2 - 4 \div \sqrt{2}x + 1 = 0 $
\item $ 4x^2 - 4x + 4 = 0 $
\end{enumerate}
\item What is the length of the arc of the sector of a circle with radius $14\mathrm{cm}$ and a central angle of $90\degree$:    
\begin{enumerate}
\item  $ 22\mathrm{cm} $                                                        
\item  $ 44\mathrm{cm} $
\item  $ 88\mathrm{cm}$                                            
\item  $ 11\mathrm{cm}$
\end{enumerate}
\item If $\triangle ABC  \triangle PQR$ with $\angle A = 32\degree $ and $\angle R = 65\degree$, then the measure of $\angle B$ is?:
\begin{enumerate}
\item $ 32\degree $                                                
\item $ 65\degree $                                            
\item $ 83\degree $                                
\item $ 97\degree $
\end{enumerate}
\begin{figure}[!ht]                              
\centering                              
\includegraphics[width=\columnwidth]{image1.jpg}      
\label{fig:enter-label}                                
\end{figure}                                            
\item If $p$ and $q$ are natural numbers and $p$ is a multiple of $q$, then what is the HCF of $p$ and $q$?:
\begin{enumerate}
\item $ pq $  
\item $ p $                                        
\item $ q $                              
\item $ p+q $
\end{enumerate}
\item  The coordinates of vertex A of a triangle ABCD whose three vertices are given as B\brak{0,0}, C\brak{3,0}, and D\brak{0,4} are?:  
\begin{enumerate}
\item $  \brak{4,0} $                      
\item $  \brak{0,3} $                          
\item $  \brak{3,4} $                                  
\item $  \brak{5,3} $
\end{enumerate}
\begin{figure}[!ht]
\centering
\includegraphics[width=\columnwidth]{image2.jpg}      
\label{fig:enter-label}
\end{figure}]
\item If the pair of equations $3x - y + 8 = 0$ and $6x - ry + 16 = 0$ represent coincident lines, then the value of 'r' is?:                          \begin{enumerate}        
\item $ \frac{-1}{2} $                        
\item $ \frac{1}{2} $                        
\item $ -2 $      
\item $ 2 $
\end{enumerate}
\item A Bag Contains $100$ Cards Numbered $1$ to $100$ . A Card Is Drawn At Random From The Bag. What Is The Probability That The Number On The Card Is A Perfect Cube?:    
\begin{enumerate}
\item $ \frac{1}{20} $              
\item $ \frac{3}{50} $                    
\item $ \frac{1}{25} $                                                
\item $ \frac{7}{100} $
\end{enumerate}
\item The Pair Of Equations $x=a$ and $y=b$ graphically represents Lines Which Are?:    
\begin{enumerate}
\item $ parallel $                                  
\item $ intersecting at \brak{b,a} $                                
\item $ coincident $    
\item $ intersecting at \brak{a,b}) $
\end{enumerate}
\begin{figure}[h!]    
\centering                                            
\includegraphics[width=\columnwidth]{image3.jpg}    
\label{fig:image3}        
\end{figure}]
\item If one zero of the polynomial $6x^2 + 37x - \brak{k-2}$ is the reciprocal of the other,
then what is the value of k?:      
\begin{enumerate}
\item $ -4 $          
\item $ -6 $
\item $ 6 $                                          
\item $ 4 $
\end{enumerate}
\item What Is The Total Surface Area Of a Solid Hemisphere Of Diameter 'd'?:    \begin{enumerate}                          
\item $ 3\pi d^2 $                        
\item $ 2\pi d^2 $                                    
\item $ \frac{1}{2}\pi d^2 $
\item $ \frac{3}{4}\pi d^2 $
\end{enumerate}
\item If three coins are tossed simultaneously, what is the probability of getting at most one tail?:
\begin{enumerate}
\item $ \frac{3}{8} $                              
\item $ \frac{4}{8} $                            
\item $ \frac{5}{8} $  
\item $ \frac{7}{8} $    
\end{enumerate}
\item In the given figure, $DE \parallel BC$. If AD = 2 units, DB = AE =3 units and EC = x units, then the value of x is:  
\begin{enumerate}
\item $ 2 $                                                          
\item $ 3 $                    
\item $ 5 $                                      
\item $ \frac{9}{2} $
\end{enumerate}
\newpage                                                                            
\begin{figure}[h!]
\centering                                            
\includegraphics[width=\columnwidth]{image4.jpg}    
\label{fig:enter-label}                                
\end{figure}
\item The hour-hand of a clock is $6\mathrm{cm}long$ . The angle swept byit between $7:20 a.m.$ and $7:55 a.m.$ is:
\begin{enumerate}
\item $ \left(\frac{35}{4}\right)\degree $
\item $ \left(\frac{35}{2}\right)\degree $                                
\item $ 35\degree $                  
\item $ 70\degree $  
\end{enumerate}
\item The zeroes of the polynomial $p\brak{x} = x ^ 2 + 4x + 3$are given by:
\begin{enumerate}                        
\item $ \brak{1,3} $                            
\item $ \brak{-1,3} $                
\item $ \brak{1,-3} $
\item $ \brak{-1,-3} $                                
\end{enumerate}                                                  
\item In the given figure, the quadrilateral PQRS circumscribes a circleHere PA + CS is equal to:  
\begin{enumerate}
\item $ QR $
\item $ PR $                                          
\item $ PS $                                          
\item $ PQ $
\end{enumerate}  
\begin{figure}[!ht]                                    
\centering                                            
\includegraphics[width=\columnwidth]{image5.jpg}      
\label{fig:image5}                                      
\end{figure}                                            
\item If $a$ and $b$ are the zeroes of the quadratic polynomial $ p\brak x = x^2 - ax - b $ then the value of $ \alpha^
2 + \beta ^2 is $:
\begin{enumerate}
\item $ a^2-2b $                                  
\item $ a^2+2b $                                  
\item $ b^2-2a $                              
\item $ b^2+2a $  
\end{enumerate}
\item The area of the triangle formed by the line axes is: $\frac{x}{a} +\frac{y}{b} = 1$ with the coordinate axes is:  
\begin{enumerate}
\item $ ab $                                    
\item $ \frac{1}{2}ab $        
\item $ \frac{1}{4}ab $              
\item $ 2ab $
\end{enumerate}
\item In the given figure,$ AB \parallel PQ. If AB = 6\mathrm{cm}, PQ =   2\mathrm{cm} and OB = 3\mathrm{cm}$, then the length of OP is:
\begin{enumerate}
\item $ 9\mathrm{cm} $                                      
\item $ 3\mathrm{cm} $        
\item $ 4\mathrm{cm} $                                    
\item $ 1\mathrm{cm} $  
\end{enumerate}
\begin{figure}[h!]      
\centering
\includegraphics[width=\columnwidth]{image6.jpg}
\label{fig:image6}                                
\end{figure}]
\item Questions number $19$ and $20$ are Assertion and Reason based questions carrying $1$ mark each. Two statements are given, one labelled as Assertion \brak{A} and the other is labelled as Reason\brak{R}). Select the correct answer to these questions from the codes\brak{a}, \brak{b}, \brak{c} and \brak{d} as given below:
\begin{enumerate}
\item $ Both Assertion \brak{A} and Reason \brak{R} are true and Reason \brak{R} is the correct explanation of the Assertion \brak{A} $                          
\item $ Both Assertion \brak{A} and Reason \brak{R} are true,but Reason \brak{R} is not the correct explanation of the Assertion \brak{A} $  
\item $ Assertion \brak{A} is true, but Reason \brak{R}is false $  
\item $ Assertion\brak{A} is false, but Reason \brak{R} is true $
\end{enumerate}
\item \textbf{Assertion\brak{A}):} A tangent to a circle is perpendicular to the radius through the point of contact.
\textbf{Reason \brak{R}):} The lengths of tangents drawn from an external point to a circle are equal.                                                                       \item \textbf{Assertion \brak{A}):} The polynomial $p(x) = x2 + 3x + 3$ has two real zeroes.    
\textbf{Reason\brak{R}:} A quadratic polynomial can have at most two real zeroes    
\end{enumerate}      
\end{document}
                                                                                                                   
                                                                                                                   
                                                                                                                   

